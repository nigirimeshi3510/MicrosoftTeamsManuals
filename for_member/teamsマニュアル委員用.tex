\documentclass[a4paper,11pt]{jsarticle}



% 数式
\usepackage{amsmath,amsfonts}
\usepackage{bm}
% 画像
\usepackage[dvipdfmx]{graphicx}
\graphicspath{{./images/}}

\begin{document}

\title{Microsoft Teamsマニュアル}
\author{情報システム責任者 高橋 塁}
\date{\today}
\maketitle

\tableofcontents
\newpage

\section{はじめに}
今年度の富大祭運営委員会では,Microsoft Teamsを用いて情報共有や連絡を行います.\\
本マニュアルでは、Microsoft Teamsの基本的な使い方について説明します.

\section{導入の背景}
昨年度でLINE WORKSの共有フォルダがいっぱいになったため,代替の連絡ツールとしてMicrosoft Teamsを使用します.理由は,全員が大学から配布された本名のアカウントを持っているから,重要なメッセージが埋もれてしまうから,作成するグループの上限が緩いから,などです.

\section{チーム と チャット}
Microsoft teamsでは,LINEまたはLINE WORKSのように1つのタブで全ての連絡を行うのではなく,「チーム」と「チャット」の2つのタブを使い分けます.
\subsection{チーム}
チームタブでは,主に全体や各部署内の連絡に用います(全体会議や集金の告知など). \\
\begin{figure}[htbp]
  \centering
  \includegraphics[width=0.5\textwidth]{post.png}
  \caption{投稿画面}
  \label{fig:screenshot}
\end{figure}

\subsection{チャット}
チャットタブでは

\section{アンケート機能}
\subsection{簡単なアンケート}
「チャット」ではpollsアプリを用いて選択式の簡単なアンケートを作成できます.\\
\begin{figure}[htbp]
  \centering
  \includegraphics[width=0.5\textwidth]{polls.png}
  \caption{pollsで作成したアンケート}
  \label{fig:screenshot}
\end{figure}

\begin{itemize}
  \item 手順\\
  左下の「+」ボタンから「polls」を選択\\
  \item アンケート削除の方法\\
  「その他」からブラウザに飛び,左上の「Forms」を押すと自分が作成したものが一覧で出てくるので,「自分のフォーム」タブから削除できます.
\end{itemize}
\subsection{複雑なアンケート}
LINEWORKSのアンケート機能のような複雑なアンケートは「チーム」の「Forms」アプリを用いて作成します.\\

\section{トラブルシューティング}
\end{document}